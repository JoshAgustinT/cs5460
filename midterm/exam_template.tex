% Exam Template for UMTYMP and Math Department courses
%
% Using Philip Hirschhorn's exam.cls: http://www-math.mit.edu/~psh/#ExamCls
%
% run pdflatex on a finished exam at least three times to do the grading table on front page.
%
%%%%%%%%%%%%%%%%%%%%%%%%%%%%%%%%%%%%%%%%%%%%%%%%%%%%%%%%%%%%%%%%%%%%%%%%%%%%%%%%%%%%%%%%%%%%%%

% These lines can probably stay unchanged, although you can remove the last
% two packages if you're not making pictures with tikz.
\documentclass[11pt]{exam}
\RequirePackage{amssymb, amsfonts, amsmath, latexsym, verbatim, xspace, setspace}
\RequirePackage{tikz, pgflibraryplotmarks}

% By default LaTeX uses large margins.  This doesn't work well on exams; problems
% end up in the "middle" of the page, reducing the amount of space for students
% to work on them.
\usepackage[margin=1in]{geometry}


% Here's where you edit the Class, Exam, Date, etc.
\newcommand{\class}{ICS 143A}
\newcommand{\term}{Fall 2017}
\newcommand{\examnum}{Midterm}
\newcommand{\examdate}{11/15/2017}
\newcommand{\timelimit}{9:00am - 9:50am}

% For an exam, single spacing is most appropriate
\singlespacing
% \onehalfspacing
% \doublespacing

% For an exam, we generally want to turn off paragraph indentation
\parindent 0ex

\begin{document} 

% These commands set up the running header on the top of the exam pages
\pagestyle{head}
\firstpageheader{}{}{}
\runningheader{\class}{\examnum\ - Page \thepage\ of \numpages}{\examdate}
\runningheadrule

\begin{flushright}
\begin{tabular}{p{2.8in} r l}
\textbf{\class} & \textbf{Name (Print):} & \makebox[2in]{\hrulefill}\\
\textbf{\term} &&\\
\textbf{\examnum} &&\\
\textbf{\examdate} &&\\
\textbf{Time Limit: \timelimit} & & \\
\end{tabular}\\
\end{flushright}
\rule[1ex]{\textwidth}{.1pt}




%\begin{minipage}[t]{3.7in}
%\vspace{0pt}
\begin{itemize}

\item \textbf{Don't forget to write your name on this exam.} 

\item \textbf{This is an open book, open notes exam. But no online or 
    in-class chatting.  } 

    
\item \textbf{Ask us if you something is confusing in the questions.}

\item \textbf{Organize your work}, in a reasonably neat and coherent way, in
the space provided. Work scattered all over the page without a clear ordering will 
receive very little credit.  

\item \textbf{Mysterious or unsupported answers will not receive full
credit}.  A correct answer, unsupported by explanation will receive no credit; 
an incorrect answer supported by substantially correct explanations might still 
receive partial credit.

\item If you need more space, use the back of the pages; clearly indicate when you have done this.
\end{itemize}

%Do not write in the table to the right.
%\end{minipage}
%\hfill

%\begin{minipage}[t]{2.3in}
%\vspace{0pt}
%\cellwidth{3em}
%\gradetablestretch{2}
\vqword{Problem}
\addpoints % required here by exam.cls, even though questions haven't started yet.	
\gradetable[v]%[pages]  % Use [pages] to have grading table by page instead of question

%\end{minipage}
\newpage % End of cover page

%%%%%%%%%%%%%%%%%%%%%%%%%%%%%%%%%%%%%%%%%%%%%%%%%%%%%%%%%%%%%%%%%%%%%%%%%%%%%%%%%%%%%
%
% See http://www-math.mit.edu/~psh/#ExamCls for full documentation, but the questions
% below give an idea of how to write questions [with parts] and have the points
% tracked automatically on the cover page.
%
%
%%%%%%%%%%%%%%%%%%%%%%%%%%%%%%%%%%%%%%%%%%%%%%%%%%%%%%%%%%%%%%%%%%%%%%%%%%%%%%%%%%%%%

\begin{questions}

% Basic question
\addpoints
\question Basic page tables.


\begin{parts}
\part[10] 

Illustrate the page table used by xv6 to map the kernel into the 
virtual address space of each process (draw a page table diagram and explain the page table entries). 
%
Specifically concentrate on one entry: the entry responsible for
the translation of the first page of the kernel. 
%
Keep in mind that xv6 maps the kernel into the virtual address range starting
above the second gigabyte of virtual memory. 
%
Note, that after xv6 is done booting, it xv6 uses normal 4KB, 32bit, 2-level
page tables. 
%
You also have to recall the physicall address of the first kernel page (look at
the boot lecture or the kernel map), and the virtual address where this page is
mapped. 
%
To make the example realistic, don't forget that xv6 allocates memory for it's
page table directory and page tables from the kernel memory allocator. 


\vfill 


%\part[10] Imagine you want to construct a two-level page table but have only
%one 4K physical page for both page table directory, and all level 2 page
%tables. What kind of address spaces you will be able to consrtuct?  Draw a
%picture, provide some discussion.

\vfill
\end{parts} 

\newpage \addpoints

\question Alice works on implementing a new shell for xv6. She implements a
pipe command (e.g., ls $\vert$ wc) like this: 

\begin{verbatim} 
void
runcmd(struct cmd *cmd)
{
  ...
  switch(cmd->type){
  default:
    fprintf(stderr, "unknown runcmd\n");
    exit(-1);

  case '|': pcmd = (struct pipecmd*)cmd;
    int p[2];
    pipe(p);
    int pid = fork();
    if(pid == 0){
        //child process:left side
        close(1);
        dup(p[1]);
        close(p[1]);
        close(p[0]);
        runcmd(pcmd->left);
    }
    close(0);
    dup(p[0]);
    close(p[0]);
    close(p[1]);
    wait(NULL);
    runcmd(pcmd->right);
    break;
  } 
  ...  
} 
\end{verbatim}

\begin{parts} 

\part[5] Her implementation always waits for left side to finish, but she is
not sure if it's correct since she notices that the shell that xv6 implements
(sh.c in the xv6 source tree) launches the right side right away. Can you come
up with an example for which Alice's shell fails, while the xv6's is still
correct?  Explain your answer. 

\end{parts}

% Question with parts
\newpage
\addpoints 

\question OS isolation and protection

\begin{parts} 

\part[5] Explain the organization and memory layout of the xv6 process. Draw a
diagram. Explain which protection bits are set by the kernel and explain why
kernel does it.

\vfill

\part[5] In xv6 individual processes are isolated, specifically they cannot
access each others memory. Explain how this is implemented. 

\vfill 

\iffalse

\newpage 

\part[10] Bob is very new to xv6, he writes his first program: 

\begin{verbatim}
#include "types.h"
#include "user.h"
  
int
main(int argc, char *argv[])
{   

  char buf[512], *p = 0;
    
  memmove(p, buf, sizeof(buf));
  exit();
}
\end{verbatim}

However, when he runs it, he gets the following error
\begin{verbatim}
$ copy
pid 3 copy: trap 14 err 5 on cpu 1 eip 0x0 addr 0x11f0--kill proc
$
\end{verbatim}

He notices that he initializes *p to 0x0, and tExplain what goes wrong. 

\fi


\end{parts} 

\newpage 

\addpoints

\question OS organization. 



\begin{parts} 


\part[10] \texttt{KERNBASE} limits the amount of memory a single process can
use, which might be irritating on a machine with a full 4 GB of RAM. Would
raising \texttt{KERNBASE} allow a process to use more memory (explain your answer)? 

\vfill

\end{parts}

% If you want the total number of points for a question displayed at the top,
% as well as the number of points for each part, then you must turn off the
% point-counter or they will be double counted.
%\newpage
%\addpoints
%\question[10] Even more work.
%\noaddpoints % If you remove this line, the grading table will show 20 points 
%for this problem.
%\begin{parts} \part[5] Even more...  \vspace{4.5in} \part[5] That's clearly
%too much \end{parts}



\end{questions}
\end{document}
